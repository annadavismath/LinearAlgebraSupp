\documentclass{ximera}
%% You can put user macros here
%% However, you cannot make new environments

\listfiles

\graphicspath{
{./}
{./LTR-0070/}
{./VEC-0060/}
{./APP-0020/}
}

\usepackage{tikz}
\usepackage{tkz-euclide}
\usepackage{tikz-3dplot}
\usepackage{tikz-cd}
\usetikzlibrary{shapes.geometric}
\usetikzlibrary{arrows}
%\usetkzobj{all}
\pgfplotsset{compat=1.13} % prevents compile error.

%\renewcommand{\vec}[1]{\mathbf{#1}}
\renewcommand{\vec}{\mathbf}
\newcommand{\RR}{\mathbb{R}}
\newcommand{\dfn}{\textit}
\newcommand{\dotp}{\cdot}
\newcommand{\id}{\text{id}}
\newcommand\norm[1]{\left\lVert#1\right\rVert}

%\newcommand{\desmosThreeD}[3]{Desmos link: \url{https://www.desmos.com/3d/#1}}

%\renewcommand{\desmosThreeD}[3]{\HCode{<iframe src="https://www.desmos.com/3d/#1" width="100\%" height="#3px" frameborder=0>This browser does not support embedded elements.</iframe>}}

\newcounter{unnumbered}
\renewcommand{\theunnumbered}{}  % Redefine the counter representation to be empty

\newtheorem*{bookSection}{}
\newtheorem{general}{Generalization}
\newtheorem{initprob}{Exploration Problem}

\tikzstyle geometryDiagrams=[ultra thick,color=blue!50!black]

%\DefineVerbatimEnvironment{octave}{Verbatim}{numbers=left,frame=lines,label=Octave,labelposition=topline}



\usepackage{mathtools}

\author{}
\license{Creative Commons 4.0 By-NC-SA}
%\outcome{Compute an antiderivative using basic formulas}
\begin{document}
\begin{exercise}
True or False?  If False, you should come up with a counterexample.  If True, can you give a proof?

\begin{enumerate}
    \item The zero vector is the only vector of length 0.
    \begin{multipleChoice}
 \choice[correct]{True}
 \choice{False}
 \end{multipleChoice}
 \item If $\norm{\vec{v}-\vec{w}}=0$, then $\vec{v}=\vec{w}$.
 \begin{multipleChoice}
 \choice[correct]{True}
 \choice{False}
 \end{multipleChoice}
 \item If $\vec{v}=-\vec{v}$, then $\vec{v}=\vec{0}$.
 \begin{multipleChoice}
 \choice[correct]{True}
 \choice{False}
 \end{multipleChoice}
 \item If $\norm{\vec{v}}=\norm{\vec{w}}$, then $\vec{v}=\vec{w}$.
 \begin{multipleChoice}
 \choice{True}
 \choice[correct]{False}
 \end{multipleChoice}
 \item If $\norm{\vec{v}}=\norm{\vec{w}}$, then $\vec{v}=\pm\vec{w}$.
 \begin{multipleChoice}
 \choice{True}
 \choice[correct]{False}
 \end{multipleChoice}
 \item If $\vec{v}=t\vec{w}$ for some scalar $t$, then $\vec{v}$ and $\vec{w}$ have the same direction.
  \begin{multipleChoice}
 \choice{True}
 \choice[correct]{False}
 \end{multipleChoice}
 \item If $\vec{v}$, $\vec{w}$ and $\vec{v}+\vec{w}$ are non-zero, and $\vec{v}$ is parallel to $\vec{v}+\vec{w}$, then $\vec{v}$ and $\vec{w}$ are also parallel.
 \begin{multipleChoice}
 \choice[correct]{True}
 \choice{False}
 \end{multipleChoice}
 \item $\norm{\vec{v}+\vec{w}}=\norm{\vec{v}}+\norm{\vec{w}}$
 \begin{multipleChoice}
 \choice{True}
 \choice[correct]{False}
 \end{multipleChoice}
\end{enumerate}
 
\end{exercise}

\subsection*{Source}
[Nicholson] W. Keith Nicholson, {\it Linear Algebra with Applications}, Lyryx 2021, Open Edition, Problem 4.1.21.
\end{document}