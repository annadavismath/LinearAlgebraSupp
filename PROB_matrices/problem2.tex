\documentclass{ximera}
%% You can put user macros here
%% However, you cannot make new environments

\listfiles

\graphicspath{
{./}
{./LTR-0070/}
{./VEC-0060/}
{./APP-0020/}
}

\usepackage{tikz}
\usepackage{tkz-euclide}
\usepackage{tikz-3dplot}
\usepackage{tikz-cd}
\usetikzlibrary{shapes.geometric}
\usetikzlibrary{arrows}
%\usetkzobj{all}
\pgfplotsset{compat=1.13} % prevents compile error.

%\renewcommand{\vec}[1]{\mathbf{#1}}
\renewcommand{\vec}{\mathbf}
\newcommand{\RR}{\mathbb{R}}
\newcommand{\dfn}{\textit}
\newcommand{\dotp}{\cdot}
\newcommand{\id}{\text{id}}
\newcommand\norm[1]{\left\lVert#1\right\rVert}

%\newcommand{\desmosThreeD}[3]{Desmos link: \url{https://www.desmos.com/3d/#1}}

%\renewcommand{\desmosThreeD}[3]{\HCode{<iframe src="https://www.desmos.com/3d/#1" width="100\%" height="#3px" frameborder=0>This browser does not support embedded elements.</iframe>}}

\newcounter{unnumbered}
\renewcommand{\theunnumbered}{}  % Redefine the counter representation to be empty

\newtheorem*{bookSection}{}
\newtheorem{general}{Generalization}
\newtheorem{initprob}{Exploration Problem}

\tikzstyle geometryDiagrams=[ultra thick,color=blue!50!black]

%\DefineVerbatimEnvironment{octave}{Verbatim}{numbers=left,frame=lines,label=Octave,labelposition=topline}



\usepackage{mathtools}

\author{}
\license{Creative Commons 4.0 By-NC-SA}
%\outcome{Compute an antiderivative using basic formulas}
\begin{document}
\begin{exercise}
Consider the system of equations:
$$\begin{matrix}
      x& -&3y&+&z&=&2\\
      -3x & -&4y&+&z&= &0\\
       & &2y&-&z&=&1
    \end{matrix}$$

Let $A$ be the coefficient matrix corresponding to this system, and let $\vec{b}=\begin{bmatrix}2\\0\\1\end{bmatrix}$.

Suppose that we find that  $(1, -2, -5)$ is a unique solution to this system.  What does this tell us?  Select ALL that apply.

\begin{selectAll}
 \choice[correct]{$A$ is invertable.}
 \choice[correct]{Vector $\vec{b}$  is in the span of the columns of $A$.}
 \choice{Columns of $A$ are linearly dependent.}
 \choice[correct]{Equation $A\vec{x}=\vec{b}$ has a unique solution.}
 \choice[correct]{Vector $\vec{b}$  can be written as a linear combination of the columns of $A$.}
 \choice{$A\vec{b}=\begin{bmatrix}1\\-2\\-5\end{bmatrix}$}
 \end{selectAll}
\end{exercise}

\end{document}