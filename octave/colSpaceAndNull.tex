\documentclass{ximera}
%% You can put user macros here
%% However, you cannot make new environments

\listfiles

\graphicspath{
{./}
{./LTR-0070/}
{./VEC-0060/}
{./APP-0020/}
}

\usepackage{tikz}
\usepackage{tkz-euclide}
\usepackage{tikz-3dplot}
\usepackage{tikz-cd}
\usetikzlibrary{shapes.geometric}
\usetikzlibrary{arrows}
%\usetkzobj{all}
\pgfplotsset{compat=1.13} % prevents compile error.

%\renewcommand{\vec}[1]{\mathbf{#1}}
\renewcommand{\vec}{\mathbf}
\newcommand{\RR}{\mathbb{R}}
\newcommand{\dfn}{\textit}
\newcommand{\dotp}{\cdot}
\newcommand{\id}{\text{id}}
\newcommand\norm[1]{\left\lVert#1\right\rVert}

%\newcommand{\desmosThreeD}[3]{Desmos link: \url{https://www.desmos.com/3d/#1}}

%\renewcommand{\desmosThreeD}[3]{\HCode{<iframe src="https://www.desmos.com/3d/#1" width="100\%" height="#3px" frameborder=0>This browser does not support embedded elements.</iframe>}}

\newcounter{unnumbered}
\renewcommand{\theunnumbered}{}  % Redefine the counter representation to be empty

\newtheorem*{bookSection}{}
\newtheorem{general}{Generalization}
\newtheorem{initprob}{Exploration Problem}

\tikzstyle geometryDiagrams=[ultra thick,color=blue!50!black]

%\DefineVerbatimEnvironment{octave}{Verbatim}{numbers=left,frame=lines,label=Octave,labelposition=topline}



\usepackage{mathtools}


\title{Column Space and Null Space of a Matrix} \license{CC BY-NC-SA 4.0}
\begin{document}
\begin{abstract}
\end{abstract}
\maketitle
\section*{Column Space and Null Space of a Matrix}

The templates in this section provide code samples for finding the column space and the null space of a matrix.  Same procedures can be used for finding the image and the kernel of a linear transformation. You can access our Octave code through the link at the bottom of each template.  Feel free to modify the code and experiment to learn more!  Alternatively, go to the \href{https://sagecell.sagemath.org/}{Sage Math Cell Webpage}, copy the code below into the cell, select OCTAVE as the language, and press EVALUATE.  

\subsection*{Column Space (Image of a Linear Transformation)}

Recall that the column space of a matrix is the span of the columns of the matrix.  We have the following procedure for finding the column space.

\begin{procedure}[\ref{proc:colspace}]
Given a matrix $B$, a basis for $\mbox{col}(B)$ can be found as follows:
\begin{enumerate}
\item Find $\mbox{rref}(B)$ (or any row-echelon form $B'$ of $B$.)
\item Identify the pivot columns of $\mbox{rref}(B)$ (or $B'$).
\item The columns of $B$ corresponding to the pivot columns of $\mbox{rref}(B)$ (or $B'$) form a basis for $\mbox{col}(B)$.
\end{enumerate}
\end{procedure}

We will implement this procedure in Octave.

\begin{template}\label{temp:colSpace}
We will use the matrix from Example \ref{ex:basiscolspace}.  Let $$A=\begin{bmatrix}2&-1&1&-4&1\\1&0&3&3&0\\-2&1&-1&5&2\\4&-1&7&2&1\end{bmatrix}$$
We can use Octave to find the reduced row-echelon form of $A$ (\emph{rref} command), and identify the pivot columns visually.  We can also take advantage of the expanded version of the \emph{rref} command to identify the pivot columns automatically, as shown below.

\begin{verbatim}
% Define A
A = [2 -1 1 -4 1;
    1 0 3 3 0;
    -2 1 -1 5 2;
    4 -1 7 2 1];

% Compute rref of A
[rref_A, pivot_cols] = rref(A);

% the pivot column indices are...
pivot_cols

% Extract the pivot columns from A, using pivot column indices
colSpace_basis = A(:, pivot_cols)
\end{verbatim}

\href{https://sagecell.sagemath.org/?z=eJxtTkEKwjAQvAfyh7kULNhiqiIoHoL6Ao9SSq2JBmxTklR8vhssKOjuZWeGmdkEe6VNpyA5k9jiVCATEMgWEBvOQCMww5x2NuKsiLrAEsXILCJcgfiSGM4S7GzbD0HBOaVhdUw_xbuSU_TmYUPV2LsvqTCyE5mOvnBTbx2kD20H011Mozxqp_I85-xjfhsOz-DqJvwYPbSzLahu8Ka7_g3ljPCxrxtVnWtvPH0jJ-vvB9MX3FlRVA==&lang=octave&interacts=eJyLjgUAARUAuQ==}{Link to code}.

Compare these findings to our answer in Example \ref{ex:basiscolspace}.
\end{template}  

\subsection*{Null Space of a Matrix (Kernel of a Linear Transformation)}

Recall the following definition of the null space of a matrix.
\begin{definition}[\ref{def:nullspace}] Let $A$ be an $m\times n$ matrix.  The \dfn{null space} of $A$, denoted by $\mbox{null}(A)$, is the set of all vectors $\vec{x}$ in $\RR^n$ such that $A\vec{x}=\vec{0}$.
\end{definition}

We can use the reduced row-echelon form (\emph{rref} command) to find the null space, as shown in Example \ref{ex:dimnull}.  We will omit the details here.

As you work with Octave, you may come across the \emph{null} command.  This command finds an orthonormal basis for the null space.  An orthonormal set of vectors consists of unit vectors that are pairwise orthogonal.  We demonstrate the \emph{null} function below.

\begin{template}\label{temp:nullSpace}
    We will use the matrix from Example \ref{ex:dimnull}.  Let 
$$A=\begin{bmatrix}2&-1&1&-4&1\\1&0&3&3&0\\-2&1&-1&5&2\\4&-1&7&2&1\end{bmatrix}$$
Find $\mbox{null}(A)$.

\begin{verbatim}
% Define A
A = [2 -1 1 -4 1;
    1 0 3 3 0;
    -2 1 -1 5 2;
    4 -1 7 2 1];

% Basis for null(A)
% Octave constructs an orthonormal basis, meaning that the basis elements are pairwise orthogonal unit vectors.
B=null(A)

% To find the dimension of the null space, find the number of columns in B
[rows, cols]=size(B)

% Verify that the columns of B are unit vectors
for i=1:cols
    NormOfCol=norm(B(:,i))
end    

% Verify that the columns of B are pairwise orthogonal (if more than one column)
if cols==1
    disp('The dimension of the null space is 1.');
else    
    for i=1:cols
        counter=i+1;
            while counter <= cols
                DotProduct=dot(B(:,i),B(:,counter))
                counter=counter+1;
            end
    end
end  

\end{verbatim}

\href{https://sagecell.sagemath.org/?z=eJyNUstO5DAQvEfKP9QFkdHOIDLLaiXAhwmcFw5oL4hDSDpMS4l7ZDuM2K-nnQdvoe1Idlx2dXWXfYBLatgSNmmygcHtGqscOVYnyM_SBBo5jvFTv-NpvVrH_Ry_sJ6Qk7j8DcXvFEmTAxSlZ49GHGzfttlmEcGrKpSPhEqsD66vgkdpIS5sxYrryhb3kbVER6Vl-4CwLYMONOKgljqykeUIu5Ldnj2N_AexSu8tBzxSFcT5ozQpzIt2VL8RaKP1kLBmzeRZVL4ZgHgSfldWtHw9Zfvunlw8Uknbd9aDLYo0uXWy1zIV9HfG8z_KiknjLzlunl4Ln3maohjKfltimkR_2OSnMdPo5B814qq5kNZES7IiO13yQpOTlqTxnypfmZNxg050U2natp1pmpyHBr0x-VhEzX6XHd587xP0RvKjw4VeOLUqNVQXx89Nxaikt4Gc4R_zs5pjv-WW5n2cG7wnznEp4dpJrc_G1BImZ5ZxmqjRpo-kWXWaP2mrrSMw_OjwDIwb4og=&lang=octave&interacts=eJyLjgUAARUAuQ==}{Link to code}.
\end{template}



\end{document}