\documentclass{ximera}
%% You can put user macros here
%% However, you cannot make new environments

\listfiles

\graphicspath{
{./}
{./LTR-0070/}
{./VEC-0060/}
{./APP-0020/}
}

\usepackage{tikz}
\usepackage{tkz-euclide}
\usepackage{tikz-3dplot}
\usepackage{tikz-cd}
\usetikzlibrary{shapes.geometric}
\usetikzlibrary{arrows}
%\usetkzobj{all}
\pgfplotsset{compat=1.13} % prevents compile error.

%\renewcommand{\vec}[1]{\mathbf{#1}}
\renewcommand{\vec}{\mathbf}
\newcommand{\RR}{\mathbb{R}}
\newcommand{\dfn}{\textit}
\newcommand{\dotp}{\cdot}
\newcommand{\id}{\text{id}}
\newcommand\norm[1]{\left\lVert#1\right\rVert}

%\newcommand{\desmosThreeD}[3]{Desmos link: \url{https://www.desmos.com/3d/#1}}

%\renewcommand{\desmosThreeD}[3]{\HCode{<iframe src="https://www.desmos.com/3d/#1" width="100\%" height="#3px" frameborder=0>This browser does not support embedded elements.</iframe>}}

\newcounter{unnumbered}
\renewcommand{\theunnumbered}{}  % Redefine the counter representation to be empty

\newtheorem*{bookSection}{}
\newtheorem{general}{Generalization}
\newtheorem{initprob}{Exploration Problem}

\tikzstyle geometryDiagrams=[ultra thick,color=blue!50!black]

%\DefineVerbatimEnvironment{octave}{Verbatim}{numbers=left,frame=lines,label=Octave,labelposition=topline}



\usepackage{mathtools}

\author{}
\license{Creative Commons 4.0 By-NC-SA}
%\outcome{Compute an antiderivative using basic formulas}
\begin{document}
\begin{exercise}
Let $A=\begin{bmatrix} -6 & -76 & -16 \\ 2 & 21 & 4 \\ -2 & -64 & -17 \end{bmatrix}$.

\begin{enumerate}
    \item The vector $\vec{v}_1=\begin{bmatrix}-4\\1\\-3\end{bmatrix}$ is an eigenvector of $A$ because $A\vec{v}_1= \lambda_1 \vec{v}_1$ where $\lambda_1 = $
 \begin{multipleChoice}
 \choice{-3}
 \choice{-1}
 \choice[correct]{2}
 \end{multipleChoice}

\item The vector $\vec{v}_2=\begin{bmatrix}8\\-2\\7\end{bmatrix}$ is an eigenvector of $A$ because $A\vec{v}_2= \lambda_2 \vec{v}_2$ where $\lambda_2 = $
 \begin{multipleChoice}
 \choice{-3}
 \choice[correct]{-1}
 \choice{2}
 \end{multipleChoice}

\item The vector $\vec{v}_3=\begin{bmatrix}4\\-1\\4\end{bmatrix}$ is an eigenvector of $A$ because $A\vec{v}_3= \lambda_3 \vec{v}_3$ where $\lambda_3 = $
 \begin{multipleChoice}
 \choice[correct]{-3}
 \choice{-1}
 \choice{2}
 \end{multipleChoice}

\item Since the Power Method converges to the dominant eigenvector, it can be used to approximate which eigenvector of $A$?
 \begin{multipleChoice}
 \choice{$\vec{v}_1$}
 \choice{$\vec{v}_2$}
 \choice[correct]{$\vec{v}_3$}
 \end{multipleChoice}
\end{enumerate}


.

 \end{exercise}
 
\end{document}