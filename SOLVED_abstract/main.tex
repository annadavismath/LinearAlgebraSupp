\documentclass{ximera}
%% You can put user macros here
%% However, you cannot make new environments

\listfiles

\graphicspath{
{./}
{./LTR-0070/}
{./VEC-0060/}
{./APP-0020/}
}

\usepackage{tikz}
\usepackage{tkz-euclide}
\usepackage{tikz-3dplot}
\usepackage{tikz-cd}
\usetikzlibrary{shapes.geometric}
\usetikzlibrary{arrows}
%\usetkzobj{all}
\pgfplotsset{compat=1.13} % prevents compile error.

%\renewcommand{\vec}[1]{\mathbf{#1}}
\renewcommand{\vec}{\mathbf}
\newcommand{\RR}{\mathbb{R}}
\newcommand{\dfn}{\textit}
\newcommand{\dotp}{\cdot}
\newcommand{\id}{\text{id}}
\newcommand\norm[1]{\left\lVert#1\right\rVert}

%\newcommand{\desmosThreeD}[3]{Desmos link: \url{https://www.desmos.com/3d/#1}}

%\renewcommand{\desmosThreeD}[3]{\HCode{<iframe src="https://www.desmos.com/3d/#1" width="100\%" height="#3px" frameborder=0>This browser does not support embedded elements.</iframe>}}

\newcounter{unnumbered}
\renewcommand{\theunnumbered}{}  % Redefine the counter representation to be empty

\newtheorem*{bookSection}{}
\newtheorem{general}{Generalization}
\newtheorem{initprob}{Exploration Problem}

\tikzstyle geometryDiagrams=[ultra thick,color=blue!50!black]

%\DefineVerbatimEnvironment{octave}{Verbatim}{numbers=left,frame=lines,label=Octave,labelposition=topline}



\usepackage{mathtools}


\title{Solved Problems for Ch 9} \license{CC BY-NC-SA 4.0}

\begin{document}

\begin{abstract}
\end{abstract}
\maketitle

\section*{Solved Problems for Chapter 9}

\begin{problem}\label{prb:10.17} Let $M,N$ be subspaces of a vector space $V$ and consider $M+N$
defined as the set of all $m+n$ where $m\in M$ and $n\in N$. Show that $M+N$
is a subspace of $V$.

Click the arrow to see answer.
\begin{expandable}
    We need to show that $M+N$ is closed under addition and scalar multiplication. 
 (See Theorem \ref{th:subspacetestabstract}.)  Suppose $\vec{x}$ and $\vec{y}$ are in $M+N$.  Then $\vec{x}=m_1+n_1$ and $\vec{y}=m_2+n_2$, where $m_i$ is in $M$ and $n_i$ is in $N$.  To see that $M+N$ is closed under scalar multiplication, consider $a\vec{x}=a(m_1+n_1)=am_1+an_1$.  Since $am_1$ is in $M$, and $an_1$ is in $N$, $\vec{x}$ is in $M+N$.  Next, we show that $M+N$ is closed under vector addition. $\vec{x}+\vec{y}=(m_1+m_2)+(n_1+n_2)$.  Since $(m_1+m_2)$ is an element of $M$, and $(n_1+n_2)$ is an element of $N$ we have closure under addition.
\end{expandable}
\end{problem}

\begin{problem}\label{prb:10.18} Let $M,N$ be subspaces of a vector space $V$. Then $M\cap N$ consists
of all vectors which are in both $M$ and $N$. Show that $M\cap N$ is a
subspace of $V$.

Click the arrow to see answer.
\begin{expandable}
    If a vector is in both subspaces $M$ and $N$, then its scalar multiple must also be in both.  Same is true about the sum of two elements of $M\cap N$.
\end{expandable}
\end{problem}

\begin{problem}\label{prb:10.19} Let $M,N$ be subspaces of a vector space $\mathbb{R}^{2}.$ Then $N\cup
M$ consists of all vectors which are in either $M$ or $N$. Show that $N\cup
M $ is not necessarily a subspace of $\mathbb{R}^{2}$ by giving an example
where $N\cup M$ fails to be a subspace.

Click the arrow to see answer.
\begin{expandable}
    As our example, let $M$ to be the first quadrant, and let $N$ be the third quadrant.
\end{expandable}
\end{problem}

\begin{problem}\label{prob:lintransmultbyx1}
Define $T:\mathbb{P}^2\rightarrow\mathbb{P}^3$ by $T(p(x))=xp(x)$.  
\begin{enumerate}
    \item Find $T(-x^2+2x-4)$.
    \item Is $T$ a linear transformation?  If so, prove it.  If not, give a counterexample.
\end{enumerate}

Click the arrow to see answer.

\begin{expandable}
    \begin{enumerate}
        \item $T(-x^2+2x-4)=-x^3+2x^2-4x$
        \item $T(ap(x)+bq(x))=x(ap(x)+bq(x))=axp(x)+bxq(x)=aT(p(x))+bT(p(x))$.  $T$ is a linear transformation.
    \end{enumerate}
    
\end{expandable}

\end{problem}

\begin{problem}\label{prb:10.24} Let $p(x) = 4x^2-x$. Is $p(x)$ in
 $\mbox{span} \left( x^2+x, x^2-1, -x + 2 \right)$?  Consider the question in two different ways, then compare your work for each approach.
 \begin{enumerate}
     \item Do this directly using the definition of span (Definition \ref{def:lincombabstract}).
     \item Do this using isomorphisms.
 \end{enumerate}

Click the arrow to see answer.
\begin{expandable}
    \begin{enumerate}
        \item Apply the definition of span directly.  We are looking for coefficients $a$, $b$ and $c$ such that 
        $$ax^2+ax+bx^2-b-cx+2c=4x^2-x$$
        Collecting like terms on the left and setting their coefficients equal to their counterparts on the right gives us the following system of equations.
        $$a+b=4$$
        $$a-c=-1$$
        $$-b+2c=0$$
        This gives rise to the augmented matrix
        $$\left[
\begin{array}{rrr|r}
1 & 1 & 0 & 4 \\
1 & 0 & -1 & -1\\
0 & -1 & 2 & 0
\end{array}
\right]\rightsquigarrow \left[
\begin{array}{rrr|r}
1 & 0 & 0 & 2/3 \\
0 & 1 & 0 & 10/3\\
0 & 0 & 1 & 5/3
\end{array}
\right]$$
This shows that our system has a unique solution and gives us the specific coefficients to express $p(x)$ as a linear combination of the vectors in the given set.
\item We can also look at this problem in light of isomorphisms.  Let's start by mapping $$x^2+x\mapsto\begin{bmatrix}1\\1\\0\end{bmatrix},\quad x^2-1\mapsto\begin{bmatrix}1\\0\\-1\end{bmatrix}, \quad -x+2\mapsto\begin{bmatrix}0\\-1\\2\end{bmatrix}$$
(Why is this an isomorphism?)

Is the vector $\begin{bmatrix}4\\-1\\0\end{bmatrix}$ in the span of the three vectors above?  Set up an augmented matrix to answer this question.  Compare this matrix to the matrix in part (a).
    \end{enumerate}
\end{expandable}
\end{problem}

\begin{problem}\label{prb:10.25} Let $p(x) = - x^2 + x + 2 $. Is $p(x)$ in $\mbox{span} \left( x^2 + x + 1, 2x^2 + x \right)$?

Click the arrow to see answer.
\begin{expandable}
    $p(x)$ is not in the span of the given vectors.  The system 
    $$\left[
\begin{array}{rr|r}
1 & 0 & 2 \\
1 & 1 & 1\\
1 & 2 & -1
\end{array}
\right]$$
is infeasible.
\end{expandable}
\end{problem}

\begin{problem}\label{prb:10.28} Consider the vector space of polynomials of degree at most $2$, $\mathbb{P}_{2}$. Determine whether the following is a basis for $\mathbb{P}_{2}$.
\begin{equation*}
\left\{ x^{2}+x+1,2x^{2}+2x+1,x+1\right\}
\end{equation*}

Click the arrow to see answer.
\begin{expandable}
There is a isomorphism from $\mathbb{R}^{3}$ to $\mathbb{P}
_{2}$, defined as follows:
$$
T(\vec{e}_{1})=1,\quad T(\vec{e}_{2})=x,\quad T(\vec{e}_{3})=x^{2}
$$
Thus,
$$
T\left(\left[
\begin{array}{c}
1 \\
1 \\
1
\end{array}
\right]\right) =x^{2}+x+1,\quad T\left(\left[
\begin{array}{c}
1 \\
2 \\
2
\end{array}
\right]\right) =2x^{2}+2x+1,\quad T\left(\left[
\begin{array}{c}
1 \\
1 \\
0
\end{array}
\right]\right) =1+x
$$
It follows that if
$$
\left\{ \left[
\begin{array}{c}
1 \\
1 \\
1
\end{array}
\right] ,\left[
\begin{array}{c}
1 \\
2 \\
2
\end{array}
\right] ,\left[
\begin{array}{c}
1 \\
1 \\
0
\end{array}
\right] \right\}
$$
is a basis for $\mathbb{R}^{3},$ then the polynomials will be a basis for $
\mathbb{P}_{2}$ because they will be independent. Recall that an isomorphism
takes a linearly independent set to a linearly independent set. Also, since $
T$ is an isomorphism, it preserves all linear relations.
\end{expandable}
\end{problem}

\begin{problem}\label{prb:10.29} Find a basis in $\mathbb{P}_{2}$ for the subspace
\begin{equation*}
\mbox{span}\left( 1+x+x^{2},1+2x,1+5x-3x^{2}\right)
\end{equation*}
If the above three vectors do not yield a basis, exhibit one of them as a
linear combination of the others. 

Click the arrow to see answer.
\begin{expandable}
This is the situation in
which you have a spanning set and you want to cut it down to form a linearly
independent set which is also a spanning set. Use the same isomorphism as
above. Since $T$ is an isomorphism, it preserves all linear relations so if
such can be found in $\mathbb{R}^{3}$, the same linear relations will be
present in $\mathbb{P}_{2}$.
\end{expandable}
\end{problem}

\begin{problem}\label{prb:10.83} Define $T:\mathbb{R}^{2}\rightarrow \mathbb{R}^{3}$ as follows.
\begin{equation*}
T(\vec{x})=\left[
\begin{array}{cc}
1 & 0 \\
1 & 1 \\
0 & 1
\end{array}
\right] \vec{x}
\end{equation*}
Show that $T$ is one to one. 

Click the arrow to see answer.

\begin{expandable}
    We need to show that $T(\vec{v})=T(\vec{w})$ implies that $\vec{v}=\vec{w}$. Suppose $$\begin{bmatrix}1 & 0\\1 & 1\\0& 1\end{bmatrix}\begin{bmatrix}v_1\\v_2\end{bmatrix}=\begin{bmatrix}1 & 0\\1 & 1\\0& 1\end{bmatrix}\begin{bmatrix}w_1\\w_2\end{bmatrix}$$
    Performing matrix-vector multiplication on both sides shows that $v_1=w_1$, and $v_2=w_2$.  Therefore, $\vec{v}=\vec{w}$. 
\end{expandable}

\end{problem}

\begin{problem}\label{prb:10.83b} Define $T:\mathbb{R}^{2}\rightarrow \mathbb{R}^{3}$ as follows.
\begin{equation*}
T(\vec{x})=\left[
\begin{array}{cc}
1 & 0 \\
1 & 1 \\
0 & 1
\end{array}
\right] \vec{x}
\end{equation*}
Is $T$ onto? 

Click the arrow to see answer.

\begin{expandable}
    $T$ is not onto.  One way to show this is by finding a counter-example.  We need a vector in $\RR^3$ that is not an image of any element of $\RR^2$ under $T$.  Try $\begin{bmatrix}2\\0\\3\end{bmatrix}$.
\end{expandable}

\end{problem}

\begin{problem}\label{prob:basisWRTB}
    Find the coordinates of $\vec{v}$ with respect to the ordered basis $\mathcal{B}$ of $\mathbb{M}_{2,2}$.
    $$\vec{v}=\begin{bmatrix}1 &2\\-1&0\end{bmatrix}$$
    $$\mathcal{B}=\left\{\begin{bmatrix}1&1\\0&0\end{bmatrix}, \begin{bmatrix}1&0\\1&0\end{bmatrix}, \begin{bmatrix}0&0\\1&1\end{bmatrix}, \begin{bmatrix}1&0\\0&1\end{bmatrix}\right\}$$

    Click the arrow to see answer.

    \begin{expandable}
        We need to find coefficients $a$, $b$, $c$, and $d$ such that
        $$a\begin{bmatrix}1&1\\0&0\end{bmatrix}+ b\begin{bmatrix}1&0\\1&0\end{bmatrix}+ c\begin{bmatrix}0&0\\1&1\end{bmatrix}+ d\begin{bmatrix}1&0\\0&1\end{bmatrix}=\begin{bmatrix}1 &2\\-1&0\end{bmatrix}$$
        This leads to the following system of equations:
        $$a+b+d=1$$
        $$a=2$$
        $$b+c=-1$$
        $$c+d=0$$
        We get $a=2$, $b=-1$, $c=0$, $d=0$.  Therefore $[\vec{v}]_{\mathcal{B}}=\begin{bmatrix}2\\-1\\0\\0\end{bmatrix}$.
    \end{expandable}
\end{problem}

\begin{problem}\label{prob:chOfBasesSolved1}
    Define $T:\mathbb{P}^2\rightarrow\mathbb{P}^2$ by $T(p(x))=p(x+1)$.  (Verify that $T$ is a linear transformation.) Find the matrix of $T$ if the basis for the domain and the codomain is $\left\{1, x, x^2\right\}$.

    Click the arrow to see answer.

    \begin{expandable}
        We start with a diagram:
\begin{center}
 \begin{tikzpicture}[scale=0.8]
   \fill[blue, opacity=0.5] (1,-1) rectangle (4,3);
   \fill[orange, opacity=0.6] (5,-1) rectangle (10,3);
   
    \fill[blue, opacity=0.3] (1,4) rectangle (4,7);
   \fill[orange, opacity=0.4] (5,4) rectangle (10,7);
   
   \node[] at (1.5, 7.5)  (p2)    {$\mathbb{P}^2$};
   \node[] at (9.5, 7.5)  (r3)    {$\mathbb{P}^2$};
   
   \node[] at (1.5, -1.5)  (p2)    {$\RR^3$};
   \node[] at (9.5, -1.5)  (r3)    {$\RR^3$};
   
  
    \node[] at (1.5, 6.5)  (a)    {$1$};
     \node[] at (1.5, 0)  (b)    {$\begin{bmatrix}1\\0\\0\end{bmatrix}$};
     
    
    \node[] at (9, 6.5)  (c)    {$1$};
     \node[] at (9, 0)  (d)    {$\begin{bmatrix}1\\0\\0\end{bmatrix}$};
     
     \node[] at (3.5, 4.5)  (e)    {$x^2$};
     \node[] at (3.5, 2)  (f)    {$\begin{bmatrix}0\\0\\1\end{bmatrix}$};
     
     \node[] at (6, 4.5)  (g)    {$x^2+2x+1$};
     \node[] at (6, 2)  (h)    {$\begin{bmatrix}1\\2\\1\end{bmatrix}$};
    
    \node[] at (7.5, 5.5)  (i)    {$x+1$};
     \node[] at (7.5,1)  (j)    {$\begin{bmatrix}1\\1\\0\end{bmatrix}$};
     
     \node[] at (2.5, 5.5)  (k)    {$x$};
     \node[] at (2.5, 1)  (l)    {$\begin{bmatrix}0\\1\\0\end{bmatrix}$};
     
    
    
     \draw [->,line width=0.5pt,-stealth]  (a.east)to(c.west);
     \draw [->,line width=0.5pt,-stealth, dashed]  (b.east)to(d.west);
     
     \draw [->,line width=0.5pt,-stealth]  (a.south)to(b.north);
     \draw [->,line width=0.5pt,-stealth]  (c.south)to(d.north);
     
     \draw [->,line width=0.5pt,-stealth]  (e.east)to(g.west);
     \draw [->,line width=0.5pt,-stealth, dashed]  (f.east)to(h.west);
     
     \draw [->,line width=0.5pt,-stealth]  (e.south)to(f.north);
     \draw [->,line width=0.5pt,-stealth]  (g.south)to(h.north);
     
     \draw [->,line width=0.5pt,-stealth]  (k.east)to(i.west);
     \draw [->,line width=0.5pt,-stealth, dashed]  (l.east)to(j.west);
     
     \draw [->,line width=0.5pt,-stealth]  (k.south)to(l.north);
     \draw [->,line width=0.5pt,-stealth]  (i.south)to(j.north);
     
%T and tau arrows     
     \draw [->,line width=0.5pt,-stealth, dashed]  (2.5,-1.1)to[out=340, in=200](8.5,-1.1 );
     \draw [->,line width=0.5pt,-stealth]  (2.5,7.1)to[out=20, in=160](8.5, 7.1);
     
%Function labels
      \node[] at (5.5, 8)    {$T$};
  \end{tikzpicture}
\end{center}

Based on where the standard unit vectors of $\RR^2$ map to, we have the following matrix.
$$\begin{bmatrix}1 & 1 & 1\\0 & 1 & 2\\0 & 0 & 1\end{bmatrix}$$
    \end{expandable}
\end{problem}


\section*{Bibliography}
Some of the problems come from the end of Chapter 9 of Ken Kuttler's \href{https://open.umn.edu/opentextbooks/textbooks/a-first-course-in-linear-algebra-2017}{\it A First Course in Linear Algebra}. (CC-BY)

Ken Kuttler, {\it  A First Course in Linear Algebra}, Lyryx 2017, Open Edition, pp. 469--535.

from Section 10.1 of Keith Nicholson's \href{https://open.umn.edu/opentextbooks/textbooks/linear-algebra-with-applications}{\it Linear Algebra with Applications}. (CC-BY-NC-SA)

W. Keith Nicholson, {\it Linear Algebra with Applications}, Lyryx 2018, Open Edition, pp. 501.

\end{document}